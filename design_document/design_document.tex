\input{preamble}
\graphicspath{ {assets/} }

\titlepicture[width=0.75\textwidth]{polimi_logo}
\title{Design Document}
\subtitle{Delibrary}
\author{\href{https://github.com/lorenzofratus}{Lorenzo Fratus} - \href{https://github.com/nicheosala}{Nicolò Sala}}
\professor{Luciano Baresi}
\date{June 22, 2021}
\version{1.0}

\begin{document}

\maketitle

\tableofcontents



\chapter{Introduction}
% Funzionamento generale di Delibrary

\section{Purpose}

\section{Scope}

\section{Document structure}



\chapter{Architectural design}

\section{Backend}
% - MVC pattern: API (Swagger), Services (ExpressJS), database (KnexJS)
% - RESTful APIs using JSON files
% - Heroku and PostgreSQL
% - Swagger for defining API, link to API docs and test

\section{Frontend}
% - Flutter: external packages and their usage
% - pattern MVC: Controller (talks to server with Dio), View, Model (Immutable!)
% - pattern provider, consigliato da Google (*single source of truth*)
% - componenti e cosa fanno (a livello astratto)



\chapter{External Services}

\section{Google Books API}

\section{Compuni ITA API}



\chapter{User interface design}

\section{UX diagram}

\section{Screenshots}



\chapter{Implementation, integration and test}

\section{Integration History}
% 1. API definition
% 2. first backend implementation && first frontend implementation and comunication with Google Books API
% 3. library management
% 4. exchange management
% 5. INTANTO I TEST ...

\section{Backend Tests}

\section{Frontend Tests}



\end{document}